\documentclass[11pt]{article} % For LaTeX2e
\usepackage{rldmsubmit,palatino}
\usepackage{graphicx}
\usepackage{hyperref}

\title{A More Robust Way of Teaching Reinforcement Learning and Decision Making}

\author{
Miguel Morales\thanks{http://www.mimoralea.com} \\
Department of Computer Science \\
Georgia Institute of Technology \\
Atlanta, GA 30332 \\
\texttt{mimoralea@gatech.edu} \\
}

\newcommand{\fix}{\marginpar{FIX}}
\newcommand{\new}{\marginpar{NEW}}

\begin{document}

\maketitle

\begin{abstract}
  We propose a new way of teaching reinforcement learning and decision making
  that is designed be an improvement to traditional academic teaching. We use
  a three step approach to deliver the most complete learning experience in a
  way that engages the student and allows them to grasp the concepts regardless
  of their skill level. We present a specific way of teaching the content, a
  new and fully configured coding platform, a set of hands-on exercises and
  a group of recommended next steps for deeper learning.
\end{abstract}

\keywords{teaching tutorials jupyter intuition hands-on}
\repository{https://www.github.com/mimoralea/applied-reinforcement-learning}
\spresentation{https://youtu.be/ltjS5ktziLQ}
\lpresentation{https://youtu.be/1WjNj_JmFaE}

\acknowledgements{
  I am thankful to my mentor, Kenneth Brooks, for providing assistance when
  navigating the field of Educational Technology. Also, when giving direct,
  concise and clear feedback on how to make this project better. Thank you
  to all my peers who also provided sincere feedback throughout the semester.
  I hope to see you all enjoying our OMSCS course in Reinforcement Learning
  and Decision Making. It is a rewarding experience. Pun intended.
}

\startmain % to start the main 1-4 pages of the submission.

\section{Introduction}
Reinforcement Learning and Decision Making is a complex subject. Being the
focus of research of a variety of fields including artificial intelligence,
psychology, machine learning, operations research, control theory, animal
and human neuroscience, economics, and ethology, it is expected that the
vast amount of available information could become counterproductive.
Beginners often find themselves lost while trying to grasp the key concepts
that are truly vital for understanding. Additionally, reinforcement learning
and decision making, being a relatively new field, is often taught by
world-class researchers that frequently unintentionally omit explaining
core concepts that might seem too basic, but are as well fundamental. This
creates a gap of knowledge that, if left unfilled, causes trouble for learning
the more advanced topics. This present a challenge for sparking interest and
keeping students engaged throughout their learning experience. If the content
is not delivered correctly, the students can quickly feel confused, lost and
disengaged, and when that happens learning stops.

\section{Sparking Curiosity}

Fortunately, since reinforcement learning and decision making is studied
by fields like animal and human neuroscience, ethology, and psychology, often
the concepts can be taught on an intuitive level. The notion of learning
by interacting with an environment should be easy to understand to all of
us as this is one of the ways we learn.

We leverage this fact and use a strategy to keep readers engaged on the
material.

\subsection{Using Simple And Direct Language}

One of the important things to we accomplished is to use simple and direct
language throughout the documents. This keeps the reader engaged regardless
of their reinforcement learning knowledge level.

We carefully select words and examples that bring the concepts to a
common sense understanding so that all students can follow the initial
readings.

\subsection{Keeping A Single Narrative}

Additionally, and what was perhaps the most difficult part, we keep a single
narrative throughout the sequence of concepts being presented. The intention
here is to allow students to continue reading and use the understanding they
accumulated in previous lessons to understand the following.

The more simplistic approach is to select concepts from the entire body of
reinforcement learning and decision making and use different lessons to present
different material. However, the problem with this approach is that it does
not help the student get the full picture and the connection with other topics.
The effort to present concepts in logical sequence, although complex to define
initially, not only feel a more natural way to present beginners, but it helps
beginners stay engaged in the material.

\subsection{Showing Concepts And Their Complement}

Finally, in order to spark and maintain students' curisity on, we show the
full spectrum of a single concept. Even if just defining the opposite, we
still make an effort to mention it and briefly explain it. Often things in
life have a complementary side that when combined better show the qualities
of each other. For example, explaining deterministic actions is interesting
all by itself, but you could gain a much better understanding if I explained
them along side stochastic actions.

We paid close attention to show concepts and their complements in every
lesson. The expectation is that this would help the students have a better
sense of the full range of possibilities on any given point, keeping them
this way engaged as concepts get progressively more and more complex.

\section{Removing Friction}

Once the student's intuition is engaged and curiosity has been sparked, a
convenient way to interact with the concepts should be presented. The
friction of getting hands-on experience is one of the most difficult
barriers to break, but once this is past, the student can better
understand the concepts.

\subsection{Setting Up A Convenient Environment}
\subsection{Providing With Boilerplate Code}
\subsection{Asking For Minimal Effort}

\section{Showing Options}
Lastly, connecting to intuition and getting hands-on experience will be
futile unless the students have a new interest of exploring the field by
themselves. Therefore, showing the path for further learning is a final
and very important step.

\subsection{Assigning Relevant Readings}
\subsection{Watching Academic Lectures}
\subsection{Completing Homework and Projects}

This document is an example of \texttt{thebibliography} environment using 
in bibliography management. Three items are cited: \textit{The \LaTeX\ Companion} 
book \cite{latexcompanion}, the Einstein journal paper \cite{einstein}, and the 
Donald Knuth's website \cite{knuthwebsite}. The \LaTeX\ related items are
\cite{latexcompanion,knuthwebsite}.

\medskip
 
\begin{thebibliography}{9}
\bibitem{latexcompanion} 
Michel Goossens, Frank Mittelbach, and Alexander Samarin. 
\textit{The \LaTeX\ Companion}. 
Addison-Wesley, Reading, Massachusetts, 1993.
 
\bibitem{einstein} 
Albert Einstein. 
\textit{Zur Elektrodynamik bewegter K{\"o}rper}. (German) 
[\textit{On the electrodynamics of moving bodies}]. 
Annalen der Physik, 322(10):891–921, 1905.
 
\bibitem{knuthwebsite} 
Knuth: Computers and Typesetting,
\\\texttt{http://www-cs-faculty.stanford.edu/\~{}uno/abcde.html}
\end{thebibliography}


\end{document}
